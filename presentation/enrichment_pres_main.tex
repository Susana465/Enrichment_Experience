\documentclass[t]{beamer}

\usetheme{TuringLight}
%\usetheme{TuringDark}

% Presentation data
\subtitle{Ethical standards and reproducibility of computer models in Neurobiology}
\date{19/01/2023}
\author{Susana Roman Garcia, PhD student.}

% Uncomment any of these lines below to set custom size for each of the font sizes.
% The default value is shown in the comment.
%\setlength{\titlefontsize}{6.875\basefontsize}
%\setlength{\subtitlefontsize}{4.375\basefontsize}
%\setlength{\frametitlesize}{2.625\basefontsize}
%\setlength{\framesubtitlesize}{1.625\basefontsize}
%\setlength{\bodytextsize}{2\basefontsize}
%\setlength{\blocktitlesize}{\bodytextsize}
%\setlength{\blockbodysize}{\bodytextsize}

% Start document
\begin{document}

% Title slide (details filled from presentation data fields above)
\begin{frame}
	\titlepage
\end{frame}

% Imprint slide (e.g. about the institute / opening quote)
\begin{frame}{16 week (Sep 22 - Jan 23) project in collaboration with:}
    \hfill \break
	\includegraphics[height=0.25\paperheight,keepaspectratio]{OLS_logo.png}
        \includegraphics[height=0.25\paperheight,keepaspectratio]{uoe_logo.png}
        \includegraphics[height=0.25\paperheight,keepaspectratio]{ATI_logo.png}
\end{frame}

\begin{frame}{With special thanks to:}
    \begin{itemize}
        \item Siobhan Mackenzie Hall, OLS mentor.

        \item Melanie Stefan, David Sterratt, Nicola Romano, PhD supervisors.

        \item Claudia Fischer, Alan Turing Institute guidance.

        \item Everyone at OLS and anyone who came along the journey.
    \end{itemize}
\end{frame}

\begin{frame}{Contents}
	\tableofcontents
\end{frame}

% Section divider slide
\section{1.Project background}
\subsection{1.1. Aim and motivation of this project:}
\begin{frame}{1.1. Aim and motivation of this project:}
	\begin{block}{Offer a case study example of how to create a PhD that looks at reproducibility and ethics as part of the process, not as an add-on.}
        \hfill \break
  		\begin{itemize}    
  			\item To tackle reproducibility issues and stop wasting money, time and resources in general.
  			\item Reproducibility only makes sense if bias is accounted for too. Otherwise, oppressive biases carry on without being questioned.
                \item Bias: inclination or prejudice for or against a certain group, especially in a way considered to be unfair.
  		\end{itemize}    
	\end{block}
\end{frame}

\section{2. Process}
\subsection{2.1. Goals achieved, key understandings.}
\begin{frame}{2.1. Goals achieved, key understandings.}
	\begin{block}{Create a written guide for looking at bias and reproducibility in a PhD.}
  		\begin{itemize}    
  			\item For other people to use as an example.
  			\item Will embed with PhD thesis, in process.
  		\end{itemize}    
	\end{block}
 
	\begin{block}{Look at speciesist bias and reproducibility in Computational Neurobiology.}
  		\begin{itemize}    
  			\item Quantifying speciesist bias in literature to showcase importance.
  			\item Lots of brainstorming. Lots of hours spent deciding which questions to ask.
                \item Lots of time spent looking for templates of other people's work.
  		\end{itemize}    
	\end{block}
        \begin{block}{Publish results...contained in GitHub for now.}   
	\end{block}
\end{frame}

\subsection{2.2. So...did I accomplish the same as goals originally set?}
\begin{frame}{2.2. So...did I accomplish the same as goals originally set?}
	\begin{block}{Kind of...}
  		\begin{itemize}    
  			\item I ended up learning a lot more about how GitHub works,
  			\item About licencing my work,
                \item About making more open, accessible work,
                \item Making contacts, reading a lot.
  		\end{itemize}    
	\end{block}
\end{frame}

\section{3. Outcomes}
\subsection{3.1. Transferrable skills for future projects.}
% Skeleton double-column text slide (two text columns)
\begin{frame}{3.1. Transferrable skills for future projects.}
	\begin{columns}[T,totalwidth=\textwidth]
  		\begin{column}{0.45\textwidth}
  			\begin{block}{Project Collaboration the Turing Institute.}
    				\begin{itemize}    
    					\item Keyword extraction to create quantitative analysis of speciesist bias in Computational Neuroscience papers. 
    				\end{itemize}  
			\end{block}
  		\end{column} %
  		\begin{column}{0.6\textwidth}
  			\begin{block}{Data Hazards, Ethics and Reproducibility Symposium, 10th March (Hybrid).}
			\begin{figure}
				\vspace{-\blocktitlesize}
				\includegraphics[height=0.4\paperheight,keepaspectratio]{qrcode.png}
			\end{figure}
			\end{block}
  		\end{column}%
	\end{columns}
\end{frame}



% End slide 
\begin{frame}{Thank you for listening.}	
    \begin{itemize}
        \item Want to chat? s1350728@ed.ac.uk
        \item Come to/share the one-day symposium!
        \hfill \break
        \begin{figure}
				\vspace{-\blocktitlesize}\includegraphics[height=0.5\paperheight,keepaspectratio]{qrcode.png}
			\end{figure}
    \end{itemize}
\end{frame}

\end{document}
